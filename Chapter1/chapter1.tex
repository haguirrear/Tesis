%!TEX root = ../thesis.tex
%*******************************************************************************
%*********************************** First Chapter *****************************
%*******************************************************************************

\chapter{Planteamiento de la Problemática}  %Title of the First Chapter

\ifpdf
    \graphicspath{{Chapter1/Figs/Raster/}{Chapter1/Figs/PDF/}{Chapter1/Figs/}}
\else
    \graphicspath{{Chapter1/Figs/Vector/}{Chapter1/Figs/}}
\fi


%********************************** %First Section  **************************************
\section{Motivación General}

En un futuro, se espera que la mayoría de los autos sean autónomos, probablemente eléctricos y que el manejo del sistema de transporte este interconectado y sea inteligente. Para lograr eso se necesita un entendimiento profundo del comportamiento humano a la hora de realizar la tarea de conducir.

Este conocimiento no sólo será útil en el futuro, sino que en el presente nos ayudaría a tener vehículos que sean conducido de una manera eficiente, que generen menos emisiones dañinas para el medio ambiente, y de una manera más segura, reduciendo la cantidad de accidentes de tránsito. 

Se pueden aprovechar las tecnologías 



\section{Motivación Específica y Propuesta de Solución}
