%!TEX root = ../thesis.tex
%*******************************************************************************
%*********************************** First Chapter *****************************
%*******************************************************************************

\chapter{Planteamiento de la Problemática}  %Title of the First Chapter

\ifpdf
    \graphicspath{{Chapter1/Figs/Raster/}{Chapter1/Figs/PDF/}{Chapter1/Figs/}}
\else
    \graphicspath{{Chapter1/Figs/Vector/}{Chapter1/Figs/}}
\fi


%********************************** %First Section  **************************************
\section{Motivación General}

En un futuro, se espera que la mayoría de los autos sean autónomos, probablemente eléctricos y que el manejo del sistema de transporte este interconectado y sea inteligente. Para lograr eso se necesita un entendimiento profundo del comportamiento humano a la hora de realizar la tarea de conducir.

Este conocimiento no sólo será útil en el futuro, sino que en el presente nos ayudaría a tener vehículos que sean conducidos de una manera eficiente, que generen menos emisiones dañinas para el medio ambiente, y de una manera más segura, reduciendo la cantidad de accidentes de tránsito.

Por otro lado, actualmente no se cuenta con una forma de monitorizar el comportamiento de los conductores que trabajan en el transporte pùblico. Este sistema permitiría además el desarrollo de un sistema de calificación y un historial para el conductor, para poder incentivar o regular el comportamiento de este y de esta manera mejorar el servicio de transporte.


\section{Motivación Específica y Propuesta de Solución}

Se propone entonces el diseño de un módulo de reconocimiento de estilo de conducción de bajo costo que pueda adaptarse a cualquier vehículo para su uso en el sistema de transporte público de Lima.

Este módulo consistirá de el uso de algoritmos de reconocimiento de patrones y un sistema de feedback que le advertirá al conductor cuando este realice una maniobra agresiva que consuma energía que podría haberse ahorrado y que comprometa la seguridad vial. Además el estilo de manejo de cada conductor será monitorizado enviando toda la información a un servidor para un análisis estadístico posterior.

Este sistema no sólo permitirá la adopción de un estilo de conducciñon más eficiente y seguro por parte de los conductores del sistema de transporte público. Sino que además brindará datos e información con la que no se cuenta actualmente y que podría ser utilizada para generar otros servicios, como el el seguimiento de los buses en tiempo real por parte de los usuarios.
