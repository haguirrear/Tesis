% ************************** Thesis Introduction *****************************
\begin{introduction}
%{El mundo en el que vivimos está cada vez más interconectado. En los últimos años se ha desarrollado tecnología que permite que objetos que antes simplemente operábamos, ahora se comuniquen e interactúen con nosotros de una manera más inteligente, que nos ahorra esfuerzos y mejora la calidad y seguridad de cada uno de ellos.
%Esto se ve claramente reflejado en el área automotriz, en la que muchas iniciativas por crear un auto autónomo están siendo desarrolladas.

El transporte es una actividad cotidiana que tiene un lugar muy importante en nuestras vidas. Sin embargo, en muchas ciudades esta actividad se realiza de manera ineficiente y esto tiene como consecuencia una larga lista de problemas, como el aumento de congestión vehicular, la disminución de la seguridad vial, el aumento de emisiones de CO\textsubscript{2}, etc.

Uno de los principales motivos del transporte ineficiente es la conducta de los conductores al manejar. Según \cite{publimetro_2017} el 80\% de los peruanos conduce de manera agresiva y es el país con peor desempeño en conducción de 38 países analizados. Esta conducta agresiva esta directamente relacionada con algunos de los problemas del transporte ineficiente.

Uno de ellos es el aumento de emisiones de CO\textsubscript{2}. El transporte está liderada por el uso de combustibles fósiles y, aunque los autos eléctricos han empezado a ganar terreno en el sector automovilístico, estos aún representan tan solo el 1.34\% de las ventas totales de autos en el mundo \cite{website:EV-sales}. El transporte se registró como la fuente mas grande de emisión de CO\textsubscript{2} en el 2016 en Europa y Estados Unidos con el 27\% y 28\% de las emisiones de gases de efectos invernadero respectivamente. Entonces el transporte ya representa una gran parte de las emisiones de CO\textsubscript{2}. Sin embargo, en \cite{7919305} se demostró que un conductor agresivo puede consumir hasta 20 \% más combustible que uno calmado, este consumo extra de combustible conlleva a un incremento del 50\% en emisiones de CO\textsubscript{2}. Se encuentra entonces en el estilo de conducción la oportunidad de reducir en gran medida las emisiones de este gas de efecto invernadero.


Por otro lado, otras de las consecuencias de la conducción agresiva es la seguridad vial. En el Perú en el año 2017 murieron un total de 772 personas debido a accidentes de tránsito \cite{website:El-comercio}. Las mayores causas de los accidentes de tránsito en el Perú son el exceso de velocidad (32\%) y la imprudencia del conductor (28\%) \cite{website:PNP-accidentes}. Estas dos causas son errores humanos que están directamente relacionados con el estilo de manejo de los conductores.

 Lo expuesto anteriormente convierte al estilo de conducción de un usuario en un factor relevante para la seguridad vial y para el cuidado del medio ambiente. Por lo que la presente tesis desarrollará un sistema que sea capaz de caracterizar el estilo de conducción de un usuario para luego proporcionar un feedback que lleve al conductor a mejorar sus estilo de manejo y como consecuencia reducir las emisiones generadas, el combustible consumido e incrementar la seguridad durante la conducción.


\end{introduction}
