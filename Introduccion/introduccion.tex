% ************************** Thesis Introduction *****************************
\begin{introduction}
%{El mundo en el que vivimos está cada vez más interconectado. En los últimos años se ha desarrollado tecnología que permite que objetos que antes simplemente operábamos, ahora se comuniquen e interactúen con nosotros de una manera más inteligente, que nos ahorra esfuerzos y mejora la calidad y seguridad de cada uno de ellos.
%Esto se ve claramente reflejado en el área automotriz, en la que muchas iniciativas por crear un auto autónomo están siendo desarrolladas.

En la actualidad vivimos con una importante preocupación, la contaminación ambiental. Las consecuencias que podría generar en nuestro planeta pueden ser muy graves. Una de las principales causas de la contaminación ambiental es la emisión de gases de efecto invernadero. El CO\textsubscript{2} es uno de estos gases y es el resultado de una de las actividades más cotidianas del mundo, el transporte.

Esta actividad está liderada por el uso de combustibles fósiles y, aunque los autos eléctricos han empezado a ganar terreno en el sector automovilístico, estos aún representan tan solo el 1.34\% de las ventas totales de autos en el mundo \cite{website:EV-sales}. El transporte se registró como la fuente mas grande de emisión de CO\textsubscript{2} en el 2016 en Europa y Estados Unidos con el 27\% y 28\% de las emisiones de gases de efectos invernadero respectivamente.

Por otro lado, en el Perú la seguridad vial es un aspecto que preocupa a muchos peruanos. En el año 2017 murieron un total de 772 personas debido a accidentes de tránsito \cite{website:El-comercio}. Las mayores causas de los accidentes de tránsito en el Perú son el exceso de velocidad (32\%) y la imprudencia del conductor (28\%) \cite{website:PNP-accidentes}. Estas dos causas son errores humanos que tienen mucha relación con el estilo de manejo de los conductores. Cada conductor puede ejecutar una misma maniobra de una manera distinta (más o menos temeraria) y la temeridad en las maniobras no solo impacta en la seguridad vial, sino también en el consumo de combustible. Esto convierte al estilo de conducción de un usuario en un factor relevante para la seguridad vial y para el cuidado del medio ambiente.

La presente tesis desarrollará un sistema que sea capaz de caracterizar el estilo de conducción de un usuario y  relacione ese estilo con el consumo de combustible. Se propondrá un estilo de conducción ideal que sea el objetivo a alcanzar y se diseñará un sistema que proporcione de feedback que lleve al conductor a alcanzar el objetivo.


\end{introduction}
