% ************************** Thesis Introduction *****************************
\begin{introduction}
%{El mundo en el que vivimos está cada vez más interconectado. En los últimos años se ha desarrollado tecnología que permite que objetos que antes simplemente operábamos, ahora se comuniquen e interactúen con nosotros de una manera más inteligente, que nos ahorra esfuerzos y mejora la calidad y seguridad de cada uno de ellos.
%Esto se ve claramente reflejado en el área automotriz, en la que muchas iniciativas por crear un auto autónomo están siendo desarrolladas.

En la actualidad vivimos con una importante preocupación, la contaminación ambiental. Las consecuencias que podría generar en nuestro planeta pueden ser muy graves. Una de las principales causas de la contaminación ambiental es la emisión de gases de efecto invernadero. El CO\textsubscript{2} es uno de estos gases y está presente en una de las actividades más cotidianas del mundo, El transporte.

Esta actividad está liderada por el uso de combustibles fósiles. Y aunque los autos eléctricos han empezado a ganar terreno en el sector automovilístico, estos aún representan tan solo el 1.3\% de las ventas totales de autos en el mundo. \mynote{Aquí falta poner la referencia} El transporte se registró como la fuente mas grande de emisión de CO\textsubscript{2} en el 2016 en Europa y Estados Unidos con el 27\% y 28\% de las emisiones de gases de efectos invernadero respectivamente.

Por otro lado, en el Perú la seguridad "vial" \mynote{Revisar este término} es un aspecto que preocupa a muchos peruanos. En el año 2017 murieron un total de 772 debido a accidentes de tránsito \mynote{Citar el comercio}. Esa no es una cifra despreciable. Las mayores causas de los accidentes de tránsito son el exceso de velocidad (32\%) y la imprudencia del conductor (28\%) \mynote{citar aquí al reporte estadístico PNP}. Estas dos causas son errores humanos que tienen mucho que ver con el estilo de manejo de los conductores. Cada conductor puede ejecutar una maniobra de manera muy agresiva como calmada y la agresividad en las maniobras no solo impacta en la seguridad vial, sino también en el consumo de combustible.

La presente tesis desarrollará un sistema que sea capaz de caracterizar el estilo de conducción de un usuario y que relacione ese estilo con su consumo de combustible. Se desarrollará un estilo de conducción ideal, que no sea muy agresivo y que sea lo mas eficiente posible


\end{introduction}
