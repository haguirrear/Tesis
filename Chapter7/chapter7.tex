\chapter{Conclusiones y Recomendaciones}
\label{chap:Conclusiones}


 \graphicspath{{Chapter6/Figuras/}{Chapter6/Figs/PDF/}{Chapter4/Figs/}}


En este capítulo se exponen las conclusiones y recomendaciones generadas al finalizar el presente trabajo. En resumen, en esta tesis se realizó el diseño de un sistema de clasificación de estilo de manejo para empresas de transporte. Para esto se analizaron los tipos de algoritmos más usados, se diseñó el dispositivo que recoge los datos de manejo y se diseñó y validó el algoritmo de clasificación.

El objetivo principal de la tesis fue lograr clasificar el estilo de conducción de un usuario. Como se observó en el Capítulo \ref{chap:algoritmo}, la precisión del algoritmo tiene un margen de mejora amplio. Sin embargo, se demostró que se pueden obtener resultados aún con datos de baja calidad y que es posible clasificar el estilo de conducción de un conductor.

La clasificación de estilo de conducción es el primer paso para lograr un sistema de tránsito más seguro, ordenado y eficiente. Al lograr que las personas usen esta información para mejorar su estilo de conducción permitiría el ahorro de combustible, dinero y un aumento en la seguridad vial. En \cite{vinitsky2018benchmarks} se demostró que insertando un solo auto autónomo con un estilo de conducción adecuado en un ambiente con autos manejados por humanos, se duplicó la velocidad promedio de todos los autos. Esto significa que si las personas mejoran su estilo de conducción; el tráfico, uno de los grandes problemas en Lima, podría disminuir.

Además se tienen las siguientes recomendaciones para futuras investigaciones:
\begin{itemize}
    \itemsep0em
    \item Los datos a usar para entrenar el algoritmo deben ser generados por experimentos controlados con diferentes conductores, rutas y condiciones de manejo, para que se asegure la correcta interpretación de las relaciones de las características con las clases por parte del algoritmo.
    \item Se recomienda también, complementar el reconocimiento de estilo de conducción con tips o indicaciones para el conductor que sirvan para que puedan mejorar su estilo de conducción. Esto se puede implementar usando algoritmos como \textit{LPP} o \textit{learning path planning}.
\end{itemize}


