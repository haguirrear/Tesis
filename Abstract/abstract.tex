% ************************** Thesis Abstract *****************************
% Use `abstract' as an option in the document class to print only the titlepage and the abstract.
\begin{abstract}
El reconocimiento de patrones usando algoritmos de machine learning e inteligencia artificial ha incrementado su popularidad debido a toda la información que actualmente tenemos disponible. Estas herramientas juegan un papel muy importante en el sector de transporte debido a su capacidad de poder optimizar el uso de recursos como el combustible o la energía eléctrica.

Esta optimización tiene beneficios en muchos ámbitos. Económicamente el usar de manera más eficiente el combustible o la energía eléctrica reduce costos. Ambientalmente el uso eficiente de combustible reduce las emisiones de CO\textsubscript{2}, logrando reducir la contaminación en las ciudades. Por último, el estilo de conducción que usa eficientemente la energía esta asociado a un estilo de conducción menos agresivo y más seguro. Algo que es de vital importancia en países como el Perú en donde se producen cientos de muertes al año debido a accidentes de tránsito.

En la presente tesis, se buscará desarrollar un sistema capaz de reconocer el estilo de conducción de un chofer del transporte público de Lima, otorgándole feedback en tiempo real y monitorizándolo para que pueda mejorar su estilo de conducción y de esta manera se reduzca el costo del combustible utilizado, las emisiones generadas y se aumente la seguridad vial en la ciudad.

Este sistema será adaptable y escalable a cualquier tipo y número de vehículos, permitiendo así su implementación de una manera sencilla.
\end{abstract}
